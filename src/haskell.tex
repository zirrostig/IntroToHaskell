%%%presentation%%%\documentclass{beamer}
%%%handout%%%\documentclass[handout]{beamer}
%%%notes%%%\documentclass[handout]{beamer}

\usetheme{CambridgeUS}
\usecolortheme{seagull}
\usepackage[latin1,utf8]{inputenc}
\usepackage{verbatim}
\usepackage{hyperref}
\usepackage{pgfpages}
\usepackage{amssymb,amsmath}
\usepackage{minted} %using patched version included in dir, provides autogobble option
\usemintedstyle{tango}

%To disable double wide presentation for notes
%%%withnotes%%%\setbeameroption{show notes on second screen}

%To Generate Notes
%%%notes%%%\setbeameroption{show only notes}
%%%notes%%%\pgfpagesuselayout{2 on 1}[letterpaper, border shrink=1cm]

%To Generate handouts
%%%handout%%%\pgfpagesuselayout{2 on 1}[letterpaper,border shrink=1cm]

%
%Haskell code environment
\newminted[hsframe]{haskell}{frame=single,linenos,autogobble,xleftmargin=1cm}
\newmint[hsline]{haskell}{}
\makeatletter
\newenvironment<>{hscode}
    {\VerbatimEnvironment
     \minted@resetoptions
     \setkeys{minted@opt}{frame=single,linenos,mathescape,autogobble,xleftmargin=1cm}
     \begin{onlyenv}#1
     \begin{VerbatimOut}{\jobname.pyg}}
    {\end{VerbatimOut}
     \minted@pygmentize{haskell}
     \end{onlyenv}
     \DeleteFile{\jobname.pyg}}
\makeatother

\title{Intro to Haskell}
\author{Zachary Stigall}
\date{\today}
\institute[\the\year]{Colorado School of Mines}

\begin{document}
\begin{frame}
    \titlepage
\end{frame}

\section{Getting Started}
\subsection{Install}
\begin{frame}[t,fragile]
	\frametitle{Install}
    \begin{itemize}[<+->]
		\item Linux:
            \begin{itemize}
                \item<.-> Debian/Ubuntu: \begin{verbatim}apt-get install haskell-platform\end{verbatim}

                \item<.-> Fedora/CentOS/Redhat: \begin{verbatim}yum install haskell-platform\end{verbatim}
                \item<.-> or use justhub

                \item<.-> Arch: Install from the AUR
                    \note[item]{Arch users are entirely capable of breaking their system on their own, and
                        should know how to use the AUR}
            \end{itemize}

        \item Windows:
            \begin{itemize}
                \item<.-> Go to: \url{http://www.haskell.org/platform/windows.html} download,
                    install
            \end{itemize}

        \item Mac:
            \begin{itemize}
                \item<.-> Option 1: Go to \url{http://www.haskell.org/platform/mac.html} download,
                    install. Requires command line devel tools from XCode

                \item<.-> Option 2: MacPorts, its in there somewhere

                \item<.-> Option 3: HomeBrew, again its in there somewhere
                    \note[item]{People who know about macports and homebrew should be good enough to get this
                        it working}
            \end{itemize}
	\end{itemize}
\end{frame}

\subsection{Why the Haskell Platform}
\begin{frame}[t,fragile]
    \frametitle{Why the Haskell Platform}
    \begin{itemize}[<+->]
        \item Why not just ghc? %Making your life easier

        \item What does the Haskell Platform provide?
            \note[item]{The haskell platform provides a large set of libraries that work well together so you can
                get started quicker and easier. Think of it like Python's 'Batteries', but for haskell}

        \item Can I get away with just ghc?
            \note[item]{Yes, but it will suck, just go to Alamode to find out how much it sucks to not have
                basic libraries}
    \end{itemize}
\end{frame}

\section{Basics}
\subsection{About}
\begin{frame}[t,fragile]
    \frametitle{About Haskell}
    \begin{itemize}[<+->]
        \item Functional!\ldots Why?
            \note[item]{Its different, fun, and not always practical, but it can easily solve problems that imperative
                languages can't}

        \item Pure! but has ways of handling impure stuff also.
            \note[item]{No side-effects you don't have to worry about that Butterfly flipping a bit while
            you're here: reference: http://xkcd.com/378/}
            \note[item]{Purity also implies that everything will be immutable}

        \item Strong and Static Typing that can also be Inferred
            \note[item]{Think C/C++ but without nice implicit type casting
                But you don't have to tell haskell you are using a String or Int or whatever if you
                don't want to, ghc will figure this out}

        \item First Class Functions
            \note[item]{A function is no different than a value, you are encouraged (and need) to make
            functions that take functions as parameters, and return functions themselves}

        \item Lazy Evaluation
            \note[item]{You won't directly deal with this, but haskell doesn't actually compute anything until
            you ask for it.}

        \item Tons and Tons of syntactic sugar
            \note[item]{Learn to love operators and symbols, you will use them a lot}

        \item Haskell is Compiled, but it has an interpreter also (GHCi)
            \note[item]{GHCi is where everyone should start, learning haskell}
    \end{itemize}
\end{frame}

\subsection{GHC}
\begin{frame}[t,fragile]
    \frametitle{GHC}
    \begin{itemize}[<+->]
        \item GHC is massive, so massive I gave it its own slide
            \note[item]{Its like a 700MB executable}

        \item It is smarter than you
            \note[item]{Don't argue with it, you will lose}

        \item It knows haskell way better than you, listen to its suggestions
            \note[item]{It will actually give you suggestions on how to fix common problems or bad ideas}

        \item If it actually compiles your code, your code 99\% of the time will not crash,
            and will do exactly what you wanted
            \note[item]{No Joke}

        \item It is really, really good at what it does, Haskell is a High Performance language,
            often programs compete with C++, Java, or C in terms of speed.
    \end{itemize}
\end{frame}

\section{Coding Environment}
\subsection{GHCI}
\begin{frame}[t,fragile]
    \frametitle{Getting Started in GHCI}
    \begin{itemize}[<+->]
        \item Run by typing ghci in a terminal
        \item Try some basic math
        \item By default Prelude is imported
            \note[item]{Prelude is the most basic default functions that you will probably want and
                need for any project, there are other Preludes also}
        \item `it' can be used to reference the last returned value
        \item Useful builtins
            \begin{itemize}
                \item \begin{verbatim}:t[ype] <something>\end{verbatim}
                    \note[item]{Gives you the type definition of something}

                \item \begin{verbatim}:l[oad] <hs file>\end{verbatim}
                    \note[item]{Loads a haskell file into ghci so you can run and test code}

                \item \begin{verbatim}:r[eload]\end{verbatim}
                    \note[item]{Reloads last loaded file}

                \item \begin{verbatim}:e[dit]\end{verbatim}
                    \note[item]{Opens the last loaded file in your editor}

                \item \begin{verbatim}:set editor <executable>\end{verbatim}
                    \note[item]{Sets you editor to be whatever you tell it}

                \item \begin{verbatim}:set prompt ``<prompt string>''\end{verbatim}
                    \note[item]{Sets your prompt, by law it should be a $\lambda$ or some other haskell-ly
                        thing}

                \item \begin{verbatim}:main <args>\end{verbatim}
                    \note[item]{Runs the main function of the loaded file, with args, simulating
                        running it from your shell like normal}

                \item \begin{verbatim}:h[elp]\end{verbatim}
                    \note[item]{Displays GHCi help, telling you all about these commands and more}
            \end{itemize}
    \end{itemize}
\end{frame}

\subsection{File}
\begin{frame}[t,fragile]
    \frametitle{First File}
    \begin{overprint}
        \only<1|handout:1>{In your editor of choice create a new file, call it example.hs}

        \only<1|handout:1>{A lot of the examples are inspired by Learn You a Haskell for Great Good}

        \only<2-3|handout:1>{Functions basics}
        \begin{itemize}
            \item<2-3|handout:1> Name has to start with lowercase letter
                \note[item]<2-|handout:1>{Uppercase names are reserved for Type Constructors}
            \item<2-3|handout:1> Convention says to use camelCase, but underscores are fine too
        \end{itemize}

        \only<2|handout:1>{Create a function}
        \begin{hscode}<2|handout:1>
            twice x = 2 * x
        \end{hscode}

        \only<3|handout:1>{Add a type signature}
        \begin{hscode}<3|handout:1>
            twice :: Integer -> Integer
            twice x = 2 * x

            --Just to show a common idiom
            twice' :: Integer -> Integer
            twice' x = x + x
        \end{hscode}
            \note[item]<3-|handout:1->{Explain how the type signature works, we will get more into detail later\\}

        \only<4|handout:2>{Using functions}
        \begin{itemize}
            \item<4|handout:2> Function application looks a lot like the definition
        \end{itemize}
        \begin{hscode}<4|handout:2>
            twiceTwo :: Integer -> Integer -> Integer
            twiceTwo x y = twice x + twice y
        \end{hscode}
            \note[item]<4-|handout:2>{Have them load their code into GHCi, and try out their functions\\}
    \end{overprint}
\end{frame}

\section{The Basics}
\subsection{Control}
\begin{frame}[t,fragile]
    \frametitle{Control Structures}
    \begin{overprint}
        \only<1|handout:1>{if..then..else}
        \begin{itemize}
            \item<1|handout:1> All if statements must have both a then clause \textit{AND} an else clause
        \end{itemize}
        \begin{hscode}<1|handout:1>
            -- The 'even' function in Prelude does just this
            -- Lets through a Type Class in here as well
            isEven :: Integral a => a -> Bool
            isEven x = if x `mod` 2 == 0
                        then True
                        else False
        \end{hscode}
            \note[item]<1-|handout:1->{Welcome to Type Classes, broad overview here}
            \note[item]<1-|handout:1->{Integral is a Type of Real and Enum, and contains Int and Integer}
            \note[item]<1-|handout:1->{Relate if to the ternary operator}
            \note[itme]<1-|handout:1->{Make sure they understand this is an expression not a statement}

        \only<2|handout:2>{case \textit{expr} of \ldots}
        \begin{hscode}<2|handout:2>
            isEven :: Integral a => a -> Bool
            isEven x = case x `mod` 2 of
                0 -> True
                1 -> False
        \end{hscode}

        \only<3-7|handout:3>{Loops}
        \begin{itemize}
            \item<3-7|handout:3> You don't need them
            \item<4-7|handout:3> You don't have them
            \item<5-7|handout:3> You have to use some form of a map
                \note[item]<5-|handout:3>{In 2 frames I cover maps, folds, filters, and list comps}
                \note[item]<5-|handout:3>{Next frame is lists}
            \item<6-7|handout:3> Or a List Comprehension
            \item<7|handout:3> \ldots or use recursion
        \end{itemize}
    \end{overprint}
\end{frame}

\section{Lists}
\subsection{Intro}
\begin{frame}[t,fragile]
    \frametitle{Lists}
    \begin{overprint}
        \only<1-|handout:1->{Lists are the bread and butter of functional programming\\}
        \only<1-|handout:1->{You will use lists everywhere\\}
        \only<2-|handout:1->{Lists are homogeneous, meaning you can't mix types like you would in Python, Ruby,
            or other high level languages}
        \begin{hscode}<3-|handout:1->
            aList :: [Integer]
            aList = [1,2,3,4]
        \end{hscode}
    \end{overprint}
\end{frame}

\subsection{Building}
\begin{frame}[t,fragile]
    \frametitle{Ranges and Construction}
    \begin{overprint}
        \only<1|handout:1>{The range operator \hsline{..}}
        \begin{hscode}<1|handout:1>
            -- .. is inclusive on both ends
            >>= [1..4]
            [1,2,3,4]
            >>= [1..]
            [1,2,3,...]
            >>= [0,2..]
            [0,2,4,...]
            -- If you want to go backwards
            >>= [9,8..0]
            [9,8,7,6,5,4,3,2,1,0]
        \end{hscode}

        \only<2|handout:2>{List construction}
        \begin{hscode}<2|handout:2>
            -- : pronounced cons => prepends an item
            >>= 'a' : ['b', 'c']
            "abc"

            -- ++ concatenate
            >>= "hello" ++ " " ++ "world"
            "hello world"
        \end{hscode}
            \note[item]<2|handout:2>{Strings are really just [Char]}
    \end{overprint}
\end{frame}

\subsection{Grouping}
\begin{frame}[t,fragile]
    \frametitle{List Groups}
    \begin{hscode}<1|handout:1>
        -- head : tail
        >>= head [10..20]
        10
        >>= tail [1..5]
        [2,3,4,5]

        -- init ++ [last]
        >>= init [1..5]
        [1,2,3,4]
        >>= last [1..5]
        5
    \end{hscode}
\end{frame}

\subsection{Length}
\begin{frame}[t,fragile]
    \frametitle{List Info}
    \begin{hscode}<1|handout:1>
        -- null :: [a] -> Bool
        >>= null []
        True
        >>= null ['a'..'f']
        False

        -- length :: [a] -> Int
        >>= length []
        0
        >>= length ['0..'z']
        75
    \end{hscode}
\end{frame}

\subsection{Misc}
\begin{frame}[t,fragile]
    \frametitle{More Functions}
    \begin{overprint}
        \begin{hscode}<1|handout:1>
            >>= reverse [1..4]
            [4,3,2,1]

            >>= take 5 [10,20..]
            [10,20,30,40,50]

            >>= drop 3 [2..9]
            [5,6,7,8,9]
        \end{hscode}

        \begin{hscode}<2|handout:2>
            >>= maximum [7,2,3,10,5,9]
            10

            >>= minimum [7,2,3,10,5,9]
            2

            >>= sum [7,2,3,10,5,9]
            36

            >>= product [7,2,3,10,5,9]
            18900
        \end{hscode}
    \end{overprint}
\end{frame}

\begin{frame}[t,fragile]
    \frametitle{Check and Access}
    \begin{hscode}<1|handout:1>
        >>= 'a' `elem` "hello world"
        False

        >>= elem 'w' "hello world"
        True

        >>= "hello world" !! 4
        'o'

        >>= head $ tail $ tail $ tail $ tail "hello world"
        'o'
    \end{hscode}
\end{frame}

\section{Iterations}
\subsection{Map}
\begin{frame}[t,fragile]
    \frametitle{Map}
    \begin{overprint}
        \only<1-|handout:1->{Applies a function to each element in a list and returns the new list}
        \begin{hscode}<1-|handout:1->
            map :: (a -> b) -> [a] -> [b]
        \end{hscode}
        \begin{onlyenv}<2|handout:1>
            \begin{minted}[frame=single,linenos,autogobble,xleftmargin=1cm]{python}
                #A little python code
                for i in lst:
                    newval = func(i)
                    newlst.append(newval)

                #Or use python's map
                newlst = map(func, lst)

                #Or a list comp
                newlst = [func(i) for i in lst]
            \end{minted}
        \end{onlyenv}

        \begin{hscode}<3|handout:2>
            -- Multiplies each item in a list by 2
            timesTwo lst = map (\x -> 2 * x) lst
        \end{hscode}

        \begin{hscode}<4|handout:3>
            -- Lets Clean up our function def a little
            -- First lets get rid of the lambda

            timesTwo lst = map (* 2) lst
        \end{hscode}

        \begin{hscode}<5-6|handout:4>
            -- Lets Clean up our function def a little
            -- First lets get rid of the lambda
            -- Second lets get rid of excess 'points'
            timesTwo = map (* 2)
        \end{hscode}

        \begin{hscode}<6|handout:4>
            -- How about a list comp
            timesTwo xs = [ 2 * x | x <- xs ]
        \end{hscode}

    \end{overprint}
\end{frame}

\subsection{Filter}
\begin{frame}[t,fragile]
    \frametitle{Filter}
    \begin{overprint}
        \only<1-|handout:1->{Checks each item against some condition. True $\Rightarrow$ keep, False $\Rightarrow$ Discard}
        \begin{hscode}<1-|handout:1->
            filter :: (a -> Bool) -> [a] -> [a]
        \end{hscode}

        \begin{onlyenv}<2|handout:1>
            \begin{minted}[frame=single,linenos,autogobble,xleftmargin=1cm]{python}
                #More python
                for i in lst:
                    if func(i):
                        newlst.append(i)

                #Or python's filter
                newlst = filter(func, lst)

                #Or a list comp
                newlst = [i for i in lst if func(i)]
            \end{minted}
        \end{onlyenv}

        \begin{hscode}<3-5|handout:2>
            -- Looks a lot like map
            evens xs = filter even xs
        \end{hscode}

        \begin{hscode}<4-5|handout:2>
            -- Cleaned up a bit
            evens = filter even
        \end{hscode}

        \begin{hscode}<5|handout:2>
            -- As a list comp
            evens xs = [ x | x <- xs, even x ]
        \end{hscode}

    \end{overprint}
\end{frame}

\subsection{Folds}
\begin{frame}[t,fragile]
    \frametitle{Folds}
    \begin{overprint}
        \only<1-|handout:1->{Applies a function to each item of a list and an accumulator, returns the
            accumulator}
        \begin{hscode}<1-|handout:1->
            foldl :: (a -> b -> a) -> a -> [b] -> a
            foldr :: (a -> b -> b) -> b -> [a] -> b
        \end{hscode}

        \begin{onlyenv}<2|handout:1>
            \begin{minted}[frame=single,linenos,autogobble,xleftmargin=1cm]{python}
                #Python once more
                for i in lst:
                    accum = func(accum, i)

                #Clear as mud right?
                #Python also has a fold it is called reduce
                from functools import reduce
                reduce(func, lst)
            \end{minted}
        \end{onlyenv}

        \begin{hscode}<3|handout:2>
            -- Lets make a difference function
            -- Continually subtracts each number
            diffl x:xs = foldl (-) x xs
            -- Or foldr
            diffr xs = foldr (-) (last xs) (init xs)
        \end{hscode}
        \begin{hscode}<4|handout:3>
            -- These don't behave the same
            >>= diffl [1..10]
            -53

            >>= diffr [1..10]
            -5
        \end{hscode}
    \end{overprint}
\end{frame}

\section{The Type System}
\subsection{So Far}
\begin{frame}[t,fragile]
    \frametitle{Types}
    \begin{overprint}
        \begin{itemize}
            \item<1-|handout:1> Haskell is strongly typed
            \item<2-|handout:1> \dots And it doesn't really like type casting (you can still do it though)
            \item<3-|handout:1> So far\dots
        \end{itemize}
    \end{overprint}
\end{frame}

\subsection{Type Signatures}
\begin{frame}[t,fragile]
    \frametitle{Type Signatures}
    \begin{overprint}
        \only<1-3|handout:1>{Your understanding so far\dots}
        \begin{hscode}<2-3|handout:1>
            funcName :: ArgT1 -> ArgT2 -> ArgT3 -> ReturnT
            -- More appropriately
            funcName :: a -> b -> c -> d
        \end{hscode}
        \only<3-|handout:1-2>{You are limiting yourself, and preventing Haskell from doing what it's
            good at, Currying}
        \only<4-7|handout:2>{There is a reason we use arrows for all args and return}
        \only<5-7|handout:2>{The return is not necessarily the last item}
        \begin{hscode}<6|handout:2>
            func1 :: Int a => a -> a -> a -> a
            func2 :: Int a => a -> a -> a
            func3 :: Int a => a -> a
            func4 :: Int a => a
        \end{hscode}

        \begin{hscode}<7-|handout:2>
            func1 :: Int a => a -> a -> a -> a
            func2 :: Int a => a -> a -> a
            func3 :: Int a => a -> a
            func4 :: Int a => a

            --Lets define these a bit
            func1 a b c = a + b + c
            func2 a b = func1 10 a b
            func3 a = func2 20 a
            func4 = func3 30
        \end{hscode}

        \only<8-|handout:3>{Another way to look at it}
        \only<9-|handout:3>{Given something, returns what is left}
            \note<9->{well actually it is what is on the right}
        \only<10-|handout:3>{So given a function that takes at most 3 values}
        \begin{hscode}<10-|handout:3>
            func1 :: a -> b -> c -> d
        \end{hscode}

        \only<11-|handout:3>{Give it one value, and you get back a new function}
        \begin{hscode}<11-|handout:3>
            func2 :: b -> c -> d
            func2 a = func1 a
        \end{hscode}
            \note<11->{Now is a really good time to talk about lambda calculus}
            \note<11->{Lambda calculus' functions always take 1 value, but can return functions}
            \note<11->{Make them ask questions, make sure they understand}
    \end{overprint}
\end{frame}

%This section needs work
\section{IO and Monads}
\subsection{FirstExe}
\begin{frame}[t,fragile]
    \frametitle{Creating your first executable}
    \begin{overprint}
        \begin{hscode}<1-2|handout:1>
            main = do
              putStr "What is your name? "
              user <- getLine
              putStr "Hi "
              putStrLn user
        \end{hscode}

        \begin{onlyenv}<2|handout:1>
            To Run: \\
            \begin{verbatim}runhaskell <yourfile>\end{verbatim}}
            Or: \\
            \begin{verbatim}ghc <yourfile> -o <exename>; ./<exename>\end{verbatim}
        \end{onlyenv}

        \begin{hscode}<3|handout:2>
            -- Alternate way if you prefer 
            --   braces and semicolons
            main = do {
              putStr "What is your name? ";
                user <- getLine;
                  putStr "Hi ";
                    putStrLn user;
            }
            -- The excess indetation is to show
            --   that with this notation haskell
            --   ignores whitespace
        \end{hscode}

        \begin{hscode}<4|handout:3>
            -- Alternate way using Monad operators
            -- Note: This is technically one line
            --   you are allowed to play with whitespace some
            main = putStr "What is your name? " >>
                   getLine >>=
                   putStrLn . (++) "Hi "
        \end{hscode}
    \end{overprint}
\end{frame}

\begin{frame}[t,fragile]
    \frametitle{Your first Monad}
    \begin{overprint}
        \only<1-2|handout:1>{You just did it, congrats\\}
        \only<2-|handout:1>{We are not creating monads today, that is beyond the scope of this presentation\\}
        \only<3-|handout:1>{Monads you will use as a haskell programmer, and not realize\\}
        \begin{itemize}
            \item<4-> IO
            \item<5-> Maybe
            \item<6-> Either
            \item<7-> List
            \item<8-> Many More
        \end{itemize}

    \end{overprint}
\end{frame}


\section{Closing}
\begin{frame}[t,fragile]
    \frametitle{Questions and Resources}
    \begin{itemize}
        \item<1-|handout:1-> Book/Web: \href{http://learnyouahaskell.com/}{Learn You a Haskell For Great Good}
        \item<1-|handout:1-> Book/Web: \href{http://book.realworldhaskell.org/}{Real World Haskell}
        \item<1-|handout:1-> Web:      \href{https://www.fpcomplete.com/}{School of Haskell}
    \end{itemize}

    \only<2>{\huge{Questions?}}
\end{frame}
\end{document}
